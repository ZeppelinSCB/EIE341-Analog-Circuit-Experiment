\section{The Common\-Collector BJT Amplifier}

\subsection{Experiment Design}
    \subsubsection{Background}
    An Amlifier is

    And a Common\-Collector BJT Amplifier have the characters

    It can be used of

    \subsubsection{Propose}
    \begin{itemize}
        \item Measure the quiescent-point of an emitter follwer(CC Amplifier)
        \item Ecaluate the small-signal amplification function of an emitter follower
    \end{itemize}

\subsection{Experiment Design}
    \subsubsection{Materials}
        In this experiment, we will use the following components:
        \begin{itemize}
            \item 2N3904\_1
            \item Capacitors
            \item Resistors
            \item DC Power Supply
            \item Function Generator
            \item Oscilloscope
            \item Digital Multimeter
        \end{itemize}
    \subsubsection{Circuit Diagram}
        The circuit diagram of the Common\-Collector BJT Amplifier is shown in Figure \ref{fig:CC_Amplifier}.
        \begin{figure}[H]
            \centering
            \begin{circuitikz}
                \draw (0,0) node[npn](npn1){};
                \draw (npn1.E) to[R,l=$R_E$] ++(0,-2) node[ground]{};
                \draw (npn1.C) to[R,l=$R_C$] ++(0,2) node[above]{Vcc};
                \draw (npn1.B) to[short,-o] ++(-1,0) node[left]{Vin};
                \draw (npn1.C) to[short,-o] ++(1,0) node[right]{Vout};
            \end{circuitikz}
            \begin{circuitikz}
                \draw (0,0) 
                    to [R,l=$R_{s}$] (2,0)
                    to [C, l=$10\mu F$] (4,0);
            \end{circuitikz}
            \caption{Common\-Collector BJT Amplifier}
            \begin{circuitikz}[american]

                % Input Voltage and Resistor
                \draw (0,0) to[sinusoidal voltage source, l=$V_i$] (0,-2)
                      -- (2,-2) to[R, l=$R_S$] (2,0) -- (4,0);
                \draw (2,-2) -- (2,-3) node[ground]{};
                
                % Coupling Capacitor
                \draw (4,0) to[C, l=$10\mu F$] (6,0);
                
                % Base Circuit and Resistor
                \draw (6,0) -- (6,1) to[R, l=$R_B$] (6,3) -- (4,3) node[vcc]{$+12V$};
                
                % Transistor
                \draw (6,0) -- (8,0) node[npn, anchor=B, label=B] (Q1) {};
                
                % Emitter Resistor and Capacitor
                \draw (Q1.E) -- (8,-2) to[R, l=$R_E$] (8,-3) node[ground]{};
                \draw (8,-2) to[C, l=$10\mu F$] (10,-2);
                
                % Load Resistor
                \draw (10,-2) -- (10,0) to[R, l=$R_L$] (10,3) node[vcc]{$+12V$};
                
                % Output Voltage
                \draw (10,0) -- (12,0) node[right]{$V_o$};
                
                \end{circuitikz}
            
            \label{fig:CC_Amplifier}
        \end{figure}
    \subsubsection{Theoretical Analysis}

\subsection{Experiment record}
    \subsubsection{Data Recorded}
    \subsubsection{Data Analysis}
    \begin{enumerate}
        %\item \textbf{Real Output Impedance}\par
        %    First, we assumet the open-circuit voltage gain is $V_{oc}$, then, we can write the output voltage with load as the following:
        %    \begin{equation}
        %            V_{L1} = V_{OC} \cdot \frac{R_{L1}}{Z_{out} + R_{L1}}
        %    \end{equation}
        %    \begin{equation}
        %            V_{L2} = V_{OC} \cdot \frac{R_{L2}}{Z_{out} + R_{L2}}
        %    \end{equation}
        %    Then, we can get emilate the $V_{oc}$, and find the expression of the output impedance as the following:
        %    \begin{equation}
        %        Z_{out} = \frac{R_{L1} \cdot R_{L2} \cdot (V_{L1} - V_{L2})}{V_{L2} \cdot R_{L1} - V_{L1} \cdot R_{L2}}
        %    \end{equation}

        %    Substitude the value of $V_{L1}$ and $V_{L2}$ for resistance $300k\Omega$ and $1k\Omega$ we can calculate the real output impedance of the circuit:

        %    \begin{equation}
        %        Z_{out} = \frac{300k\Omega \cdot 1k\Omega \cdot (0.588V - 0.568V)}
        %                       {0.568V \cdot300k\Omega -0.588V \cdot 1k\Omega}
        %                = 
        %    \end{equation}
        \item A
    \end{enumerate}
\subsection{Experiment Conclusion}
    \subsubsection{Discussion}
    \subsubsection{Conclusion}
